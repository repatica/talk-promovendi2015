\documentclass{beamer}
\usepackage[latin1]{inputenc}
\usetheme{Warsaw}
\title{Worms and Caterpillars}
\author{Nela Leki\'c}
\date{\today}
\begin{document}

\begin{frame}
\titlepage
\end{frame}


\begin{frame}{Introduction}
Based on paper
\begin{center}
\textbf{Phylogenetic partitions and ternary permutation constraint satisfaction problems on two linear orders}
\end{center}
by 
\begin{center}
\textbf{Leo van Iersel, Steven Kelk, Nela Leki\'c, Simone Linz}
\end{center}
\end{frame}

\begin{frame}{Introduction}
\begin{itemize}
 \item A ternary Permutation-CSP is specified by a subset $\Pi$ of the symmetric group $S_3$ 
 \item An instance of such a problem consists of a set of variables (or taxa) $X$ and a multiset of constraints
 \item The constraints are ordered triples of distinct variables of $X$
 \item The objective is to answer weather all constraints given by an instance of $\Pi$ can be satisfied by a number of linear orders
\end{itemize}
\end{frame}

\begin{frame}{The complexity of the 11 ternary permutations CSPs}
\begin{table}
\begin{center}
\begin{tabular}{l l c c}
	 &				&1LO	&2LO \\ \hline
 $\Pi_0 \text{ (linear ordering)}$ & 123 				& P 	& NPC \\
 $\Pi_1$ & 123, 132 			& P 	& NPC \\
 $\Pi_2$ & 123, 213, 231 		& P 	& P \\
 $\Pi_3$ & 123, 231, 312, 321 		& P 	& P \\
 $\Pi_4$ & 123, 231			& NPC 	& NPC \\
 $\Pi_5 \text{ (betweenness)}$ & 123, 321			& NPC 	& NPC \\
 $\Pi_6$ & 123, 132, 231		& NPC 	& NPC \\
 $\Pi_7 \text{ (circular ordering)}$ & 123, 231, 312 		& NPC 	& P \\
 $\Pi_8$ & $S_3 \setminus$ 123, 231  	& NPC 	& P \\
 $\Pi_9 \text{ (non-betweenness)} $ & $S_3 \setminus$ 123, 321  	& NPC 	& NPC \\
 $\Pi_{10}$ & $S_3 \setminus$ 123 	& NPC 	& P 
\end{tabular} 
\end{center}
\end{table} 
\end{frame}


\begin{frame}{An example}
 Problem $2-\Pi_5$ consists of constraints of form $(123,321)$. To say that a linear order satisfies a $\Pi_5$-constraint $c=(abc)$ is to say that in that linear order either triple $abc$ or triple $cba$ is satisfied.
\end{frame}


\begin{frame}{Some phylogenetic background}
\begin{itemize}
 \item A binary rooted phylogenetic on $n$ leaves
 \item Caterpillar
 \item Triplet
 \item When a tree is \textit{displayed} by another tree
\end{itemize}

 
\end{frame}


\begin{frame}{$k$-Caterpillar Compatibility}
\begin{itemize}
\item[Instance] A set $R$ of rooted triplets.
\item[Question] Do there exist at most $k$ caterpillars such that each element in $R$ is displayed by at least one such caterpillar?
\end{itemize}
\end{frame}


\end{document}